% Debugged LaTeX Template

\documentclass[11pt]{article}
\usepackage[margin=1in,a4paper]{geometry}
\usepackage{amsmath,amssymb,amsthm}
\usepackage{graphicx}
\usepackage{hyperref}
\usepackage{xcolor}
\usepackage{listings}
\usepackage{caption}
\usepackage{biblatex}
\addbibresource{reference.bib}

\title{Progress Report for Autonomous Parking with DRL}
\author{Zheyuan Wu,Yuchen Wu}
\date{\today}

\begin{document}
\maketitle

\begin{abstract}
The main problem of the project is to train an agent that controls a vehicle to park in the desired spot while avoiding collision with obstacles. This problem arises from the rapid development of autonomous vehicles. Compared to parking manually, autonomous parking not only saves time, but also achieves more compact parking spaces: it can significantly boost the operation efficiency of both private cars and commercial trucks. 

We aim to train an agent that is able to park the vehicle in the target spot from any feasible starting position, under time or space constraints. We approach the problem by modeling it as an environment with continuous state space (position, angle, etc.) and discrete action space (steering, direction, etc.); then we will attempt to train the agent with two algorithms: DQN and PPO. While traditional studies often focus on the parking of cars, we will attempt to park trailer trucks, which have more complex mechanics than cars and are more common in industrial contexts. A real-world application of the problem would be autonomous parking of trailer trucks to facilitate efficient cargo loading and unloading. To start with, we aim to train the agent to back a trailer truck straight into a parking spot that is directly behind it (this is easy for cars, but tricky for trailer trucks).

Different from these existing works, we will attempt to solve the parking problem with tighter time and space constraints, in order to maximize the time and space efficiency of parking.

\end{abstract}

%%%%%%%%% BODY TEXT
\section{Project Overview}
We aim to leverage the existing deep reinforcement learning models to solve this problem and test various algorithms to train the agent to back a trailer truck straight into a parking spot with constrained open spaces.

For the first stage of the experiments, we tested the DQN algorithm to train the agent to back a trailer truck straight into a parking spot with constrained open spaces.

\section{Team Member Roles/Tasks}
\label{sec:roles}

\subsection{Yuchen Wu}

\begin{enumerate}

\item Build the environment via gymnasium.
\item Test basic DQN algorithm
\item Writing the paper.
\end{enumerate}

\subsection{Zheyuan Wu}

\begin{enumerate}
\item Provides supporting functions and code reviews for the environment
\item Test DDPG algorithm
\item Maintaining documentation and consistent environment
\item Writing the paper and create visualization
\end{enumerate}


\section{Collaboration Strategy}
Zheyuan has built a GitHub repository, and Yuchen build the environment via gymnasium.

\section{Initial Results}

We have implemented the environment for trailer trucks using pygame and gymnasium. We have tested the DQN algorithm to train the agent to back a trailer truck straight into a parking spot with constrained open spaces.

We use stable-baselines3 \cite{stable-baselines3} as our main framework for deep reinforcement learning. 

\section{Current Concerns and Questions}

\printbibliography

\end{document}